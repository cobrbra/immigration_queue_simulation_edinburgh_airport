\documentclass[10pt]{article}
\textwidth 16.5cm
\textheight 26cm
\oddsidemargin 0pt
\topmargin -3cm
% \usepackage{epsf}

% Draft watermark
\usepackage[
stamp = false,
firstpageonly = true
]{draftwatermark}

% Font and formatting
\usepackage[default]{lato}
\usepackage[skip=3pt]{parskip}
% \usepackage{titlesec}
% \titlespacing{\paragraph}{0pt}{*1}{*2}

 \usepackage[compact]{titlesec} 

% Writing maths
\usepackage{
    amsmath, % aligns, equations, etc.
    amsfonts, % blackboard bold, etc.
    bbm, % blackboard bold for numbers.
    pifont, % for planes
}

% Figures
\usepackage{graphicx}
\usepackage{floatrow}
\floatsetup[table]{capposition=top}
\newcommand*{\figuretitle}[1]{%
    {\centering%   <--------  will only affect the title because of the grouping (by the
    \textbf{#1}%              braces before \centering and behind \medskip). If you remove
    \par\medskip}%            these braces the whole body of a {figure} env will be centered.
}

% Boxes
\usepackage{enumitem}
\usepackage{tcolorbox}
\definecolor{edi-dark-purple}{rgb}{0.4882812,0.046875,0.4296875}
\definecolor{edi-light-purple}{rgb}{0.9453125,0.8359375,0.9140625}

% References
\usepackage[
    colorlinks,
    linkcolor=black,
    citecolor=edi-dark-purple, 
    urlcolor=edi-dark-purple,
    breaklinks = true
]{hyperref}
\usepackage{xurl}


% Acronyms
\usepackage[acronym, toc]{glossaries-extra}

\setabbreviationstyle[acronym]{long-short}
\glssetcategoryattribute{acronym}{nohyperfirst}{true}
\renewcommand*{\glsdonohyperlink}[2]{%
 {\glsxtrprotectlinks \glsdohypertarget{#1}{#2}}}

 \newacronym{bfo}{BFO}{Border Force Officer}
 \newacronym[plural=SLAs, firstplural=Service Level Agreements]{sla}{SLA}{Service Level Agreement}
 \newacronym[plural=eGates, firstplural=electronic passport Gates]{egate}{eGate}{electronic passport Gate}
 \newacronym[plural=KPIs, firstplural=Key Performance Indicators]{kpi}{KPI}{Key Performance Indicator}

% Inline comments from Jacob and Bella
\usepackage{xcolor}
\usepackage[draft,inline,nomargin,index]{fixme}
\fxsetup{theme=color,mode=multiuser}
\FXRegisterAuthor{jb}{ajb}{\color{blue} JB}
\FXRegisterAuthor{bd}{abd}{\color{red} BD}

\title{Planning for Future Demand on Border Operations\\ at Edinburgh Airport (High-Level Summary)}
 \author{Isabella Deutsch and Jacob R. Bradley}
 \date{}


\begin{document}
\maketitle
\thispagestyle{empty}

\vspace{-15pt}
\paragraph{Introduction}
Edinburgh Airport has requested an investigation into the construction of new \glspl{egate} to support border force operations. This document provides an overview of results.
\begin{tcolorbox}[
colframe=edi-dark-purple,
colback=edi-light-purple,
fonttitle=\bfseries,
title = {Report Highlights}]
\begin{enumerate}[itemsep=-1ex]
   \vspace{-1.5mm}
    \item[\ding{40}] \textbf{Building 13 \glspl{egate} over five years can keep \gls{egate} queue \glsxtrshortpl{kpi} within appropriate levels.}\\
    \vspace{-1.5mm}
    \item[\ding{40}] \textbf{Regardless of new \glsxtrshortpl{egate}, long desk queues pose a serious risk until \glsxtrshort{egate} usage increases.}\\
    \vspace{-1.5mm}
    \item[\ding{40}] \textbf{Effective \glsxtrshort{egate} usage may be achieved by expanded eligibility or by increased uptake.}\\
    \vspace{-1.5mm}
    \item[\ding{40}] \textbf{Lower-than-recommended \glsxtrshort{egate} construction will impact queue lengths before wait times.}
    \vspace{-1.5mm}
\end{enumerate}
\end{tcolorbox}
\paragraph{Recommendations}
From a baseline of 10 \glsxtrshortpl{egate} currently available, we propose a phased 130\% increase to 23 \glsxtrshortpl{egate} spread over the five year period. This sees the construction of 5 \glsxtrshortpl{egate} for 2023, 4 for 2024, 2 for 2025, and a further 2 for 2026 to adequactly manage passenger flow.  We also strongly recommend that the airport pursue measures to increase \glsxtrshort{egate} usage via eligibility and uptake, in line with the 2025 UK Border Strategy. 


\paragraph{Contributions}
We developed a simulation framework for analysing the impact of \glsxtrshort{egate} construction on border check queues. In addition, we considered changes to \glsxtrshort{egate} eligibility, uptake, and allocation of immigration hall space. Based on our passenger processing model we were able to explore the airport's chosen \glsxtrshortpl{kpi} over a large number of simulations. We give an overview of our central recommendation scenario in the figure below. % Figure~\ref{fig:core_rec_fig}.
\begin{figure}[!h]
    \centering
    \includegraphics[width=\textwidth]{figures/core_rec_fig.png}
     %\caption{Recommended \glsxtrshort{egate} construction schedule, alongside usage assumptions and \glsxtrshort{kpi} summaries.} \label{fig:core_rec_fig}
\end{figure}

\vspace{-5pt}

We predict that excellent \gls{egate} wait times are achievable, and that the main pressure on immigration services comes via queue lengths demanding regular use of the airport's overflow and contingency spaces. This is especially pressing for desk queues until the airport is able to improve overall \glsxtrshort{egate} usage. 

In line with the project brief, we provide robustness analyses in order to understand the implications of varying scenarios on our core conclusions. To allow further investigation, we present a bespoke Shiny Application. 

\begin{tcolorbox}[
colframe=edi-dark-purple,
colback=edi-light-purple,
fonttitle=\bfseries,
title = {Use our Shiny Application to interactively explore demand scenarios}]
\vspace{-1.5mm}
\begin{itemize}
\item[\ding{40}] URL: \url{https://jacob-bradley.shinyapps.io/shiny/}
\item[\ding{40}] Username: \texttt{edi-airport} \quad Password: \texttt{flowhost}
\vspace{-1.5mm}
\end{itemize}
\end{tcolorbox}


\end{document}