\documentclass[12pt]{article}
\textwidth 16.5cm
\textheight 23.5cm
\oddsidemargin 0pt
\topmargin -2cm
% \usepackage{epsf}

% Writing maths
\usepackage{
    amsmath, % aligns, equations, etc.
    amsfonts, % blackboard bold, etc.
    bbm, % blackboard bold for numbers.
}

% Figures
\usepackage{graphicx}

% References
\usepackage{hyperref}

% References 
\usepackage{natbib}
\bibliographystyle{natbib}
\setcitestyle{authoryear, open={(},close={)}}

% Inline comments from Jacob and Bella
\usepackage{xcolor}
\usepackage[draft,inline,nomargin,index]{fixme}
\fxsetup{theme=color,mode=multiuser}
\FXRegisterAuthor{jb}{ajb}{\color{blue} JB}
\FXRegisterAuthor{bd}{abd}{\color{red} BD}

\title{Simulations of Immigration Queues at Edinburgh Airport: Report}
 \author{Isabella Deustch, Jacob R. Bradley
 \\ \emph{School of Mathematics, University of Edinburgh}}

\begin{document}
\maketitle

\section{Introduction}
This document contains the results of Jacob and Bella's project to submit to the modelling competition. It should be ten pages or fewer. \jbnote{including references?}

At present, we show some results derived from \texttt{R}'s \texttt{mtcars} dataset, to show how a coding and writing workflow integrating with GitHub, RStudio, and Zotero might work.

\section{Methods}
\subsection{GitHub}
The code for this project is held in a \href{https://github.com/cobrbra/immigration_queue_simulation_edinburgh_airport}{GitHub repository}. This stores and tracks the history of all the data, code, and docs associated with the project, but shouldn't be used directly for editing anything aside from in an emergency! Instead, coding should be done in RStudio and writing should be done in Overleaf. In GitHub, you can see the entire project folder layout (Figure~\ref{fig:github_overview}).

\begin{figure}[htbp]
    \centering
    \includegraphics[width=4in]{figures/github_overview.png}
    \caption{GitHub view of project folder.}
    \label{fig:github_overview}
\end{figure}

The ``Code'' tab will be useful for syncing GitHub with RStudio (see below). Syncing with Overleaf should already have been done from here.

\subsection{RStudio}
\subsection{Overleaf}
\subsection{Zotero}

\section{Results}

\begin{figure}[htbp]
    \centering
    \includegraphics[width=4in]{figures/mt_cars_summary.png}
    \caption{Example figure generated from mtcars workflow.}
    \label{fig:mtcars}
\end{figure}

\begin{figure}[htbp]
    \centering
    \includegraphics[width=4in]{figures/mpg_model_actual_vs_predicted.png}
    \caption{Example figure generated from a model of mtcars data.}
    \label{fig:mtcars_model}
\end{figure}

% latex table generated in R 4.2.2 by xtable 1.8-4 package
% Tue Feb 21 14:59:41 2023
\begin{table}[ht]
\centering
\begin{tabular}{rrrrrrrrrrrr}
  \hline
 & mpg & cyl & disp & hp & drat & wt & qsec & vs & am & gear & carb \\ 
  \hline
1 & 18.70 & 8.00 & 360.00 & 175.00 & 3.15 & 3.44 & 17.02 & 0.00 & 0.00 & 3.00 & 2.00 \\ 
  2 & 18.10 & 6.00 & 225.00 & 105.00 & 2.76 & 3.46 & 20.22 & 1.00 & 0.00 & 3.00 & 1.00 \\ 
  3 & 14.30 & 8.00 & 360.00 & 245.00 & 3.21 & 3.57 & 15.84 & 0.00 & 0.00 & 3.00 & 4.00 \\ 
  4 & 19.20 & 6.00 & 167.60 & 123.00 & 3.92 & 3.44 & 18.30 & 1.00 & 0.00 & 4.00 & 4.00 \\ 
  5 & 17.80 & 6.00 & 167.60 & 123.00 & 3.92 & 3.44 & 18.90 & 1.00 & 0.00 & 4.00 & 4.00 \\ 
  6 & 16.40 & 8.00 & 275.80 & 180.00 & 3.07 & 4.07 & 17.40 & 0.00 & 0.00 & 3.00 & 3.00 \\ 
  7 & 17.30 & 8.00 & 275.80 & 180.00 & 3.07 & 3.73 & 17.60 & 0.00 & 0.00 & 3.00 & 3.00 \\ 
  8 & 15.20 & 8.00 & 275.80 & 180.00 & 3.07 & 3.78 & 18.00 & 0.00 & 0.00 & 3.00 & 3.00 \\ 
  9 & 10.40 & 8.00 & 472.00 & 205.00 & 2.93 & 5.25 & 17.98 & 0.00 & 0.00 & 3.00 & 4.00 \\ 
  10 & 10.40 & 8.00 & 460.00 & 215.00 & 3.00 & 5.42 & 17.82 & 0.00 & 0.00 & 3.00 & 4.00 \\ 
  11 & 14.70 & 8.00 & 440.00 & 230.00 & 3.23 & 5.34 & 17.42 & 0.00 & 0.00 & 3.00 & 4.00 \\ 
  12 & 15.50 & 8.00 & 318.00 & 150.00 & 2.76 & 3.52 & 16.87 & 0.00 & 0.00 & 3.00 & 2.00 \\ 
  13 & 15.20 & 8.00 & 304.00 & 150.00 & 3.15 & 3.44 & 17.30 & 0.00 & 0.00 & 3.00 & 2.00 \\ 
  14 & 13.30 & 8.00 & 350.00 & 245.00 & 3.73 & 3.84 & 15.41 & 0.00 & 0.00 & 3.00 & 4.00 \\ 
  15 & 19.20 & 8.00 & 400.00 & 175.00 & 3.08 & 3.85 & 17.05 & 0.00 & 0.00 & 3.00 & 2.00 \\ 
  16 & 15.80 & 8.00 & 351.00 & 264.00 & 4.22 & 3.17 & 14.50 & 0.00 & 1.00 & 5.00 & 4.00 \\ 
  17 & 19.70 & 6.00 & 145.00 & 175.00 & 3.62 & 2.77 & 15.50 & 0.00 & 1.00 & 5.00 & 6.00 \\ 
  18 & 15.00 & 8.00 & 301.00 & 335.00 & 3.54 & 3.57 & 14.60 & 0.00 & 1.00 & 5.00 & 8.00 \\ 
   \hline
\end{tabular}
\caption{MTCars dataset restricted to observations with <20 miles per gallon \label{tab:mtcars_full}} 
\end{table}


\section{Discussion}
\section{Conclusion}
% \bibliography{references.bib}
\end{document}